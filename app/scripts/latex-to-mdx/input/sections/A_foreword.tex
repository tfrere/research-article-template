\section*{Foreword}

Robotics is an inherently multidisciplinary field, and is not witnessing unprecedented advancements since its inception in the 1960s.
Yet, more than sixty years after the debut of Unimate, robots have still not fully integrated into the rich, unstructured, and dynamic world we humans inhabit.
Over the decades, numerous disciplines have shown immense promise in tackling the challenges of creating autonomous systems.
This tutorial takes a clear stance in the debate on whether modern Machine 
Learning can play a pivotal role in the development of 
autonomous robot systems: we believe this to be the case.

Nonetheless, we also hold that the wealth of research from both academia and industry in classical robotics over the past six decades is, simply put, too valuable to be cast aside in favor of purely learning-based methods.
However, the interplay between classical robotics and modern machine learning is still in its nascent stages, and the path to integration yet to be clearly defined.
In turn our goal here is to present what we consider to be the most relevant approaches within robot learning today, while warmly extending an invite to collaborate to expand the breadth of this work! Start contributing today \href{https://github.com/fracapuano/robot-learning-tutorial}{here}.

This tutorial\dots
\begin{itemize}
    \item Does \emph{not} aim to be a comprehensive guide to general field of robotics, manipulation or underactuated systems:~\citet{sicilianoSpringerHandbookRobotics2016} and~\citet{tedrakeRoboticManipulationPerception,tedrakeUnderactuatedRoboticsAlgorithms} do this better than we ever could.
    \item Does \emph{not} aim to be an introduction to statistical or deep learning:~\citet{shalev-shwartzUnderstandingMachineLearning2014} and~\citet{prince2023understanding} cover these subjects better than we ever could.
    \item Does \emph{not} aim to be a deep dive into Reinforcement Learning, Diffusion Models, or Flow Matching: invaluable works such as~\citet{suttonReinforcementLearningIntroduction2018},~\citet{nakkiranStepbyStepDiffusionElementary2024}, and~\citet{lipmanFlowMatchingGuide2024} do this better than we ever could.
\end{itemize}

Instead, our goal here is to provide an intuitive explanation as per why these disparate ideas have converged to form the exciting field of modern robot learning, driving the unprecedented progress we see today. 
In this spirit, we follow the adage: "a jack of all trades is a master of none, \emph{but oftentimes better than a master of one}."

We sincerely hope this tutorial serves as a valuable starting point for your journey into robot learning.